section{Desarrollo}

\subsection{Formulacion del Sistema}

Para poder armar el sistema $A.x = b $, 

\subsection{Ecuación de calor}

La matriz $A$, es una matriz cuadrada de tamaño $(m-2) x n$ donde los coeficientes de la misma quedan determinados del despeje de la siguiente ecuación de calor:

\begin{equation} \label{ecu:calor}
	\frac{\partial^2T(r,\ \theta)}{\partial r^2} + \frac{1}{r} \frac{\partial^2T(r,\ \theta)}{\partial r} + \frac{1}{r^2} \frac{\partial^2T(r,\ \theta)}{\partial\theta^2} = 0
\end{equation}

  	Luego se discretiza cada punto para poder ser procesada: 
\begin{equation} 
  	%\frac{ {t_{j-1,k}s - 2 t_{j,k} + t_{j+1,k} }{(\Delta r)^2} + \frac{t_{j,k} - t_{j-1,k}}{r \Delta r} + \frac{t_{j,k-1} - 2 t_{j,k} + t_{j,k+1} }{r^2 (\Delta \Theta)^2} = 0 
  	\frac{t_{j-1,k} - 2t_{j,k}+t_{j+1,k}}{(\Delta r)^2}+\frac{t_{j,k} - t_{j-1,k}}{r \Delta r}+ \frac{t_{j,k-1} - 2t_{j,k} + t_{j,k+1}}{r^2 (\Delta \Theta)^2} = 0  	
\end{equation}  	
Para finalmente reescribir la ecuación como una combinación lineal de cinco temperaturas:

\begin{equation} \label{ec:discretizada}
	t_{j-1,k} (\frac{1}{(\Delta r)^{2}} + \frac{-1}{r \Delta r}) + t_{j,k} (\frac{-2}{(\Delta r)^2} + \frac{1}{r \Delta r} + \frac{-2}{(r \Delta \Theta)^{2}}) + t_{j+1,k} \frac{1}{(\Delta r)^{2}} + t_{j,k-1} \frac{1}{(r \Delta \Theta)^{2}} + t_{j,k+1} \frac{1}{(r\Delta \Theta)^{2}} = 0
\end{equation}  

De la ecuación \ref{ec:discretizada} manera quedan determinados cada uno de los coeficientes que aparecen en la matriz: \\
\\
\(
\alpha_r = \frac{1}{(\Delta r)^{2}} + \frac{-1}{r \Delta r} \\
\beta_r = \frac{-2}{(\Delta r)^2} + \frac{1}{r \Delta r} + \frac{-2}{(r \Delta \Theta)^{2}} \\
\gamma_r = \frac{1}{(\Delta r)^{2}} \\ 
\chi_r = \frac{1}{(r \Delta \Theta)^{2}}
\) \\\\

Finalmente re
 \label{ecuación}
