%&pdflatex
\documentclass[12pt]{article}
\usepackage[margin=1.0in]{geometry}

\usepackage{g-util}
\usepackage{amsfonts, amsmath, amssymb}
%\usepackage[xetex]{xcolor}

\setcounter{MaxMatrixCols}{12}
\newcommand{\Gpmatrix}[1]{\ensuremath{\begin{pmatrix} #1 \end{pmatrix}}}
\newcommand{\Gbmatrix}[1]{\ensuremath{\begin{bmatrix} #1 \end{bmatrix}}}
\newcommand{\sub}[3]{\ensuremath{#1_{#2,#3}}}
\newcommand{\supra}[2]{\ensuremath{#1^#2}}

\title{Métodos Numéricos}
\author{F. Galileo Cappella Lewi}
\date{1c2022}

\begin{document}

\maketitle

\section{Introducción}

%TODO: Contar el problema que nos presentaron
En general vamos a estar trabajando con \(m+1\) radios, y \(n\) ángulos.

\section{Formulación del Sistema}
 
\paragraph{} Partimos de la ecuación de Laplace \\
\[
  0 = \frac{\partial^2T(r,\ \theta)}{\partial r^2} + \frac{1}{r} \frac{\partial^2T(r,\ \theta)}{\partial r} + \frac{1}{r^2} \frac{\partial^2T(r,\ \theta)}{\partial\theta^2}
\]
Y la discretizamos \\ %TODO: Pasos y justificación en el medio?
\[
0 = \frac{\sub{t}{j-1}{k} - 2\sub{t}{j}{k} + \sub{t}{j+1}{k}}{(\Delta r)^2} + \frac{\sub{t}{j}{k} - \sub{t}{j-1}{k}}{r \Delta r} + \frac{\sub{t}{j}{k-1} - 2\sub{t}{j}{k} + \sub{t}{j}{k+1}}{r^2 (\Delta \Theta)^2}
\]
Podemos reescribir la ecuación como una combinación lineal de cinco temperaturas \\
\begin{align} %TODO: Más chiquita
0 = \sub{t}{j-1}{k} (\frac{1}{(\Delta r)^{2}} + \frac{-1}{r \Delta r}) + \sub{t}{j}{k} (\frac{-2}{(\Delta r)^2} + \frac{1}{r \Delta r} + \frac{-2}{(r \Delta \Theta)^{2}}) + \sub{t}{j+1}{k} \frac{1}{(\Delta r)^{2}} + \sub{t}{j}{k-1} \frac{1}{(r \Delta \Theta)^{2}} + \sub{t}{j}{k+1} \frac{1}{(r\Delta \Theta)^{2}}
\end{align}
Para simplificar la notación vamos a nombrar cada coeficiente \\ %TODO: Renombrar coeficientes para que tengan más sentido
\(
\tab \alpha_r = \frac{1}{(\Delta r)^{2}} + \frac{-1}{r \Delta r} \\
\tab \beta_r = \frac{-2}{(\Delta r)^2} + \frac{1}{r \Delta r} + \frac{-2}{(r \Delta \Theta)^{2}} \\
\tab \gamma_r = \frac{1}{(\Delta r)^{2}} \\ 
\tab \chi_r = \frac{1}{(r \Delta \Theta)^{2}}
\) \\\\
Resaltamos cuatro formas diferentes de escribir la ecuación \\
\begin{align}
  \alpha \sub{t}{j-1}{k} + \beta \sub{t}{j}{k} + \gamma \sub{t}{j+1}{k} + \chi \sub{t}{j}{k-1} + \chi \sub{t}{j}{k+1} = 0 \\
  \beta \sub{t}{j}{k} + \gamma \sub{t}{j+1}{k} + \chi \sub{t}{j}{k-1} + \chi \sub{t}{j}{k+1} = -\alpha \sub{t}{j-1}{k} \\
  \alpha \sub{t}{j-1}{k} + \beta \sub{t}{j}{k} + \chi \sub{t}{j}{k-1} + \chi \sub{t}{j}{k+1} = -\gamma \sub{t}{j+1}{k} \\
  \beta \sub{t}{j}{k} + \chi \sub{t}{j}{k-1} + \chi \sub{t}{j}{k+1} = -\alpha \sub{t}{j-1}{k} - \gamma \sub{t}{j+1}{k}
\end{align}
\\
\paragraph{} Por lo que podemos armar un sistema de ecuaciones de la forma \(Ax = b\) donde cada fila del sistema se corresponde a una instancia de la ecuación de Laplace discretizada. \\
Pero qué forma tiene cada instancia depende de si la temperatura del radio anterior o la del siguiente también son incógnitas (por lo que tienen su coeficiente en la matriz), o si son datos medidos (por lo que cambian al resultado): \\
\tab El primer y último radios (\(\sub{t}{1}{k}\) y \(\sub{t}{m+1}{k}\)) son datos medidos, por lo que no hace falta estimarlos con una ecuación. \\
\tab Como las temperaturas del radio anterior las conocemos (porque son las temperaturas internas medidas), las primeras \(n\) filas (que se corresponden con el radio \(\sub{t}{2}{k}\)) van a tener la forma de la ecuación (3). \\ %TODO: Hyperlink
\tab Luego, las siguientes \((m-2)n\) filas (que son todos los otros radios excepto el \(\sub{t}{m}{k}\)) tienen tanto al radio anteriro como el siguiente incógnitas, por lo que tienen la forma de la ecuación (2). \\ %TODO: Hyperlink
\tab Y las últimas \(n\) filas (radio \(\sub{t}{m}{k}\)) tienen la forma de la ecuación (4). \\ %TODO: Hyperlink
Entonces nos queda una matriz \(A \in \mathbb{R}^{((m-1)n \times (m-1)n)}\) armada de la forma explicada recién, el vector de incógntias \(x \in \mathbb{R}^{(m-1)n}\) con los radios "estirados" uno arriba del otro, y el vector de resultados \(b \in \mathbb{R}^{(m-1)n}\) que tiene las temperaturas medidas en las primeras y últimas \(n\) filas, y el resto 0's.

\paragraph{} El caso de un sistema con 3 radios es especial, ya que para el radio \(\sub{t}{2}{k}\) (que es el vector de incógnitas completo) conocemos tanto las temperaturas del radio anterior como del siguiente. Sólo en este caso tenemos la ecuación (5).

\subsection{Justificación no-pivoteo} 
%TODO: Submatrices no singulares

\subsection{Ejemplos}

\paragraph{3 radios y 4 ángulos:} \ \\

\(
\Gpmatrix{
  \beta & \chi & 0 & \chi \\
  \chi & \beta & \chi & 0 \\
  0 & \chi & \beta & \chi \\
  \chi & 0 & \chi & \beta \\
} \cdot \Gpmatrix{
  \sub{t}{2}{1} \\
  \sub{t}{2}{2} \\
  \sub{t}{2}{3} \\
  \sub{t}{2}{4} \\
} = \Gpmatrix{
  -\alpha\sub{t}{1}{1} - \gamma\sub{t}{3}{1} \\
  -\alpha\sub{t}{1}{2} - \gamma\sub{t}{3}{2} \\
  -\alpha\sub{t}{1}{3} - \gamma\sub{t}{3}{3} \\
  -\alpha\sub{t}{1}{4} - \gamma\sub{t}{3}{4} \\
}
\)

\paragraph{4 radios y 4 ángulos:} \ \\

\(
\Gpmatrix{
  \beta & \chi & 0 & \chi & -\gamma & 0 & 0 & 0 \\
  \chi & \beta & \chi & 0 & 0 & -\gamma & 0 & 0 \\
  0 & \chi & \beta & \chi & 0 & 0 & -\gamma & 0 \\
  \chi & 0 & \chi & \beta & 0 & 0 & 0 & -\gamma \\
  -\alpha & 0 & 0 & 0 & \beta & \chi & 0 & \chi \\
  0 & -\alpha & 0 & 0 & \chi & \beta & \chi & 0 \\
  0 & 0 & -\alpha & 0 & 0 & \chi & \beta & \chi \\
  0 & 0 & 0 & -\alpha & \chi & 0 & \chi & \beta \\
} \cdot \Gpmatrix{
  \sub{t}{2}{1} \\
  \sub{t}{2}{2} \\
  \sub{t}{2}{3} \\
  \sub{t}{2}{4} \\
  \sub{t}{3}{1} \\
  \sub{t}{3}{2} \\
  \sub{t}{3}{3} \\
  \sub{t}{3}{4} \\
} = \Gpmatrix{
  -\alpha\sub{t}{1}{1} \\
  -\alpha\sub{t}{1}{2} \\
  -\alpha\sub{t}{1}{3} \\
  -\alpha\sub{t}{1}{4} \\
  -\gamma\sub{t}{4}{1} \\
  -\gamma\sub{t}{4}{2} \\
  -\gamma\sub{t}{4}{3} \\
  -\gamma\sub{t}{4}{4} \\
}
\)

\paragraph{5 radios y 4 ángulos:} \ \\

\(
\Gpmatrix{
  \beta & \chi & 0 & \chi & -\gamma & 0 & 0 & 0 & 0 & 0 & 0 & 0 \\
  \chi & \beta & \chi & 0 & 0 & -\gamma & 0 & 0 & 0 & 0 & 0 & 0 \\
  0 & \chi & \beta & \chi & 0 & 0 & -\gamma & 0 & 0 & 0 & 0 & 0 \\
  \chi & 0 & \chi & \beta & 0 & 0 & 0 & -\gamma & 0 & 0 & 0 & 0 \\
  -\alpha & 0 & 0 & 0 & \beta & \chi & 0 & \chi & -\gamma & 0 & 0 & 0 \\
  0 & -\alpha & 0 & 0 & \chi & \beta & \chi & 0 & 0 & -\gamma & 0 & 0 \\
  0 & 0 & -\alpha & 0 & 0 & \chi & \beta & \chi & 0 & 0 & -\gamma & 0 \\
  0 & 0 & 0 & -\alpha & \chi & 0 & \chi & \beta & 0 & 0 & 0 & -\gamma \\
  0 & 0 & 0 & 0 & -\alpha & 0 & 0 & 0 & \beta & \chi & 0 & \chi \\
  0 & 0 & 0 & 0 & 0 & -\alpha & 0 & 0 & \chi & \beta & \chi & 0 \\
  0 & 0 & 0 & 0 & 0 & 0 & -\alpha & 0 & 0 & \chi & \beta & \chi \\
  0 & 0 & 0 & 0 & 0 & 0 & 0 & -\alpha & \chi & 0 & \chi & \beta \\
} \cdot \Gpmatrix{
  \sub{t}{2}{1} \\
  \sub{t}{2}{2} \\
  \sub{t}{2}{3} \\
  \sub{t}{2}{4} \\
  \sub{t}{3}{1} \\
  \sub{t}{3}{2} \\
  \sub{t}{3}{3} \\
  \sub{t}{3}{4} \\
  \sub{t}{4}{1} \\
  \sub{t}{4}{2} \\
  \sub{t}{4}{3} \\
  \sub{t}{4}{4} \\
} = \Gpmatrix{
  -\alpha\sub{t}{1}{1} \\
  -\alpha\sub{t}{1}{2} \\
  -\alpha\sub{t}{1}{3} \\
  -\alpha\sub{t}{1}{4} \\
  0 \\
  0 \\
  0 \\
  0 \\
  -\gamma\sub{t}{5}{1} \\
  -\gamma\sub{t}{5}{2} \\
  -\gamma\sub{t}{5}{3} \\
  -\gamma\sub{t}{5}{4} \\
}
\)
%\subsection{Modelado}

%\subsubsection{Matriz en "banda"}
%%TODO

\section{Estimando la isoterma}

\paragraph{} Un dato importante que nos interesa es encontrar la posición de la isoterma de 500ºC. Esta es un círculo dentro de la pared del horno todo a la misma temperatura. \\
Queremos poder calcularla porque si está "muy cerca" (ver sección 2.2.1) la estructura del horno estaría en riesgo. %TODO: Hyperlink

%TODO: Diagrama mostrando la isoterma

\paragraph{} Para encontarla buscamos dos puntos dentro de un mísmo ángulo que "rodeen" la temperatura buscada, y luego aproximamos linealmente la posición donde se encuentra entre estos dos puntos.

%TODO: Diagrama mostrando la isoterma estimada entre dos puntos

\subsection{Midiendo la Peligrosidad}

Una vez encontrada la isoterma, es importante decidir si el horno está en un estado peligroso o seguro. Para ello %TODO: Si llega al borde se rompe %TODO: Explicar la idea de medir qué pasa si cambia en 1 la temperatura en el interior, la "velocidad" del peligro, si estamos a d grados de que se rompa %TODO: Elejir cuánto es un d aceptable

\section{Resultados}

%TODO: Medir tiempo con unos ejemplos, gráficos y detalles

\section{Conclusión}

%TODO: Qué iriía acá?

\end{document}

%TODO: Math formatting
