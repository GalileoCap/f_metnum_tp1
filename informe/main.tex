\documentclass[12pt]{article}
\usepackage[margin=1.0in]{geometry}

\usepackage{g-util}
\usepackage{amsfonts, amsmath, amssymb}
\usepackage[xetex]{xcolor}

\newcommand{\sub}[3]{\ensuremath{#1_{#2,#3}}}
\newcommand{\supra}[2]{\ensuremath{#1^#2}}

\title{Métodos Numéricos}
\author{F. Galileo Cappella Lewi}
\date{1c2022}

\begin{document}

\maketitle

\section{Planteo del problema}

\section{Formulación del sistema}

\noindent Partiendo de las ecuaciones (1)-(6) tenemos: \\ \(
\tab 0 = \frac{\sub{t}{j-1}{k} - 2\sub{t}{j}{k} + \sub{t}{j+1}{k}}{(\Delta r)^2} + \frac{\sub{t}{j}{k} - \sub{t}{j-1}{k}}{r \Delta r} + \frac{\sub{t}{j}{k-1} - 2\sub{t}{j}{k} + \sub{t}{j}{k+1}}{r^2 (\Delta \Theta)^2} \\
\tab \phantom{0} = \sub{t}{j-1}{k} (\frac{1}{(\Delta r)^{2}} + \frac{-1}{r \Delta r}) + \sub{t}{j}{k} (\frac{-2}{(\Delta r)^2} + \frac{1}{r \Delta r} + \frac{-2}{(r \Delta \Theta)^{2}}) + \sub{t}{j+1}{k} \frac{1}{(\Delta r)^{2}} + \sub{t}{j}{k-1} \frac{1}{(r \Delta \Theta)^{2}} + \sub{t}{j}{k+1} \frac{1}{(r\Delta \Theta)^{2}} \) \\
Para simplificar la notación tomo: \\ \(
\tab \alpha = \frac{1}{(\Delta r)^{2}} + \frac{-1}{r \Delta r} \\
\tab \beta = \frac{-2}{(\Delta r)^2} + \frac{1}{r \Delta r} + \frac{-2}{(r \Delta \Theta)^{2}} \\
\tab \gamma = \frac{1}{(\Delta r)^{2}} \\ 
\tab \chi = \frac{1}{(r \Delta \Theta)^{2}}
\) \\
Por lo que tengo: \\
\[\alpha \sub{t}{j-1}{k} + \beta \sub{t}{j}{k} + \gamma \sub{t}{j+1}{k} + \chi \sub{t}{j}{k-1} + \chi \sub{t}{j}{k+1} = 0 \iff\]
\[\beta \sub{t}{j}{k} + \gamma \sub{t}{j+1}{k} + \chi \sub{t}{j}{k-1} + \chi \sub{t}{j}{k+1} = -\alpha \sub{t}{j-1}{k} \iff\]
\[\alpha \sub{t}{j-1}{k} + \beta \sub{t}{j}{k} + \chi \sub{t}{j}{k-1} + \chi \sub{t}{j}{k+1} = -\gamma \sub{t}{j+1}{k} \iff\]
\[\beta \sub{t}{j}{k} + \chi \sub{t}{j}{k-1} + \chi \sub{t}{j}{k+1} = -\alpha \sub{t}{j-1}{k} - \gamma \sub{t}{j+1}{k}\]

\subsection{Generalización}

\section{Modelado}

\subsection{"Matriz en banda"}

%TODO:
% Partiendo de las ecuaciones lo separo en t * coeficiente
% Lo "resuelvo" en función del radio anterior y sigiente
% Lo aplico:
%   2xn es trivial
%   3x4 ejemplo
%   4x4 confirmo
%   mxn en general
% Aprovechar que tiene forma de "banda" para reducir cuentas y espacio

\end{document}
